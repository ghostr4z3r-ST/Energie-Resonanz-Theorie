Wissenschaft ist der Versuch, Ordnung in das Unbekannte zu bringen.  
Sie sucht in allem Wandel nach den Konstanten, die bestehen bleiben,  
und in allem Zufall nach den Mustern, die ihn formen.  
Unter allen Disziplinen ist es die Physik,  
die diesen Anspruch am weitesten trägt:  
Sie beschreibt nicht nur, \emph{was} geschieht,  
sondern fragt, \emph{warum} es geschieht.  
\\
\noindent
Über Jahrhunderte hinweg wurde sie zum Fundament unseres Weltverständnisses.  
Sie zerlegte die Natur in Gesetze,  
maß Kräfte, Energien, Teilchen und Felder –  
und formte daraus ein Bild, das von beeindruckender Präzision ist.  
Doch trotz aller Erfolge blieb eines unvollendet:  
die Vereinigung ihrer eigenen Grundlagen.
\\
\noindent
Einstein erkannte, dass Raum und Zeit keine starren Koordinaten sind,  
sondern sich durch Energie und Masse krümmen.  
Seine Relativitätstheorie machte den Raum dynamisch,  
doch die Quantenmechanik sprach eine andere Sprache.  
Sie zeigte eine Welt, in der Wahrscheinlichkeiten regieren,  
in der Teilchen Wellen sind und Beobachtung Realität formt.  
Das Doppelspaltexperiment offenbarte dieses Paradox in seiner reinsten Form:  
Ein einzelnes Teilchen erzeugt ein Interferenzmuster,  
als ob es durch beide Spalte zugleich gegangen wäre.
\\
\noindent
Zwei Theorien, beide bewährt, beide richtig –  
und doch unvereinbar in ihrem Verständnis dessen,  
was sie beschreiben.  
Raum ist für die eine Bühne der Gravitation,  
für die andere nur der Hintergrund statistischer Zustände.  
Zwischen beiden liegt die größte Leerstelle der modernen Physik.
\noindent
\begin{center}
Eine Frage blieb bis heute unbeantwortet,  
und sie ist zugleich die einfachste und tiefste von allen:  
\end{center}

\vspace{2cm}
\begin{center}
    {\huge\textbf\emph{Was ist Raum?}}
\end{center}