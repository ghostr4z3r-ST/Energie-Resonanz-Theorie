\chapter{Einleitung}

Die Frage, "Was ist Raum?", wird in dieser Arbeit nicht durch neue Postulate beantwortet, sondern durch eine bewusste Einschränkung der zugrunde liegenden Mathematik und Empirie. 
Die Energie-Resonanz-Theorie (ERT) unterscheidet sich damit grundlegend von allen bisherigen theoretischen Ansätzen: 
anstatt zusätzliche Annahmen über die Natur einzuführen, wird der Gültigkeitsbereich der Theorie durch wenige, klar definierte Bedingungen festgelegt:

\section{Methodische Grundlage}

Die Theorie darf nur innerhalb der folgenden selbstauferlegten Grenzen formuliert werden:

\begin{enumerate}
    \item \textbf{Empirischer Anschluss} – jede theoretische Aussage muss prinzipiell an realen Messwerten oder numerischen Experimenten überprüfbar sein.
    \item \textbf{Keine Irrationalität} – die Beschreibung der Realität erfolgt ausschließlich über rationale, messbare und reproduzierbare Größen.
    \item \textbf{Keine Widersprüche zu bestehenden Modellen} – neue Formulierungen dürfen keine Paradoxa oder Widersprüche erzeugen, sondern müssen etablierte Theorien als Grenzfälle enthalten.
\end{enumerate}

Diese Einschränkungen ersetzen Postulate und führen dazu, dass sich jede Erweiterung der Theorie zwangsläufig an der beobachtbaren Realität messen lassen muss. 
Die ERT ist somit keine Annahme über die Welt, sondern eine methodische Reduktion auf das, was mathematisch und empirisch zugleich konsistent bleibt.

\section{Historischer Ausgangspunkt}

Der historische Bezugspunkt dieser Herangehensweise ist das \textbf{Doppelspaltexperiment}. 
Es markiert den Moment, in dem die Trennung zwischen Wellen- und Teilchenvorstellung versagte und die Rolle des Beobachters selbst ins Zentrum der Physik rückte. 
Die Energie-Resonanz-Theorie greift dieses Experiment auf, um zu zeigen, dass Welle, Teilchen und Beobachter keine getrennten Entitäten darstellen, 
sondern Ausdruck derselben zugrundeliegenden Resonanzstruktur des Raums sind.

Aus dieser Perspektive ergibt sich der weitere Aufbau der Arbeit:
Zunächst werden die mathematischen Grundlagen formuliert, die unter diesen Bedingungen zulässig sind,
danach folgen die numerischen Untersuchungen und schließlich die Formulierung der Energie-Resonanz-Theorie.
