\chapter{Mathematischer Rahmen I}
\begin{center}
    {\textbf{Die dreidimensionale Resonanzformulierung}}
\end{center}
\label{chap:mathematischer_rahmen_I}

\section{Einleitender Ansatz}

Das klassische
\textbf{Doppelspaltexperiment}
zeigt, dass die Trennung von Welle und Teilchen
keine physikalische Grundlage besitzt, sondern eine Folge unserer Beobachtung ist.
Wird die Beobachtungsebene selbst Teil der Betrachtung,
so zeigt sich, dass jedes Teilchenverhalten durch eine überlagerte Welle beschrieben werden kann.
Daraus ergibt sich die Annahme, dass Teilchen nichts anderes sind
als stabile Resonanzknoten solcher überlagerter Wellen,
deren Energieverteilung sich exponentiell in drei Dimensionen fortsetzt.
Jede dieser Überlagerungen bildet einen Knotenpunkt,
an dem sich Energie räumlich konzentriert. Dieser Knoten entspricht dem realen, beobachtbaren Teilchen.
\\
\begin{center}
    {\large\textbf{Mathematische Einschränkungen}}
\end{center}

Zur Beschreibung dieses Phänomens darf keine zweidimensionale Projektion mehr verwendet werden.
Jede Flächenbeschreibung (Kreise, Winkel, $\pi$-Bezüge)
reduziert den Raum auf eine irrationale Näherung, die keine physikalische Existenz besitzt.
Gesucht ist daher eine Operatorform,
die Raum, Energie und Zeit vollständig rational in drei Dimensionen verbindet.
Damit wird aus der Geometrie der Fläche eine Geometrie des Volumens,
deren Eigenschaften sich durch reale Messwerte bestätigen lassen müssen.

\section{Probleme zweidimensionaler Projektionen in der Physik}

Viele etablierte Gleichungen der Physik – insbesondere die Wellengleichung in ihrer Standardform –
setzen implizit eine zweidimensionale Geometrie voraus:
\[
\frac{\partial^2 \Psi}{\partial x^2} +
\frac{\partial^2 \Psi}{\partial y^2} = -k^2 \Psi.
\]
Diese Form beschreibt eine stehende Welle in einer Ebene,
nicht im realen Raum.
\newpage
\noindent
Die Erweiterung auf drei Dimensionen durch bloße Summation
\[
\nabla^2 \Psi =
\frac{\partial^2 \Psi}{\partial x^2} +
\frac{\partial^2 \Psi}{\partial y^2} +
\frac{\partial^2 \Psi}{\partial z^2}
\]
ändert daran nichts Wesentliches,
da sie den Raum als additive Kombination von Flächen behandelt.
Erst wenn der Operator selbst räumlich gekoppelt ist,
kann er reale Resonanzverhältnisse abbilden.

\section{Herleitung des Resonanzoperators}

Der Ausgangspunkt ist die klassische Wellengleichung
\[
\nabla^2 \Psi = \frac{1}{c^2} \frac{\partial^2 \Psi}{\partial t^2},
\]
die die zeitliche Ausbreitung einer Welle mit der räumlichen Krümmung ihres Feldes verknüpft.
Für stationäre Zustände der Energie gilt, dass sich die zeitliche Ableitung
als harmonische Oszillation der Form
\[
\Psi(\mathbf{x},t) = \psi(\mathbf{x}) e^{i\omega t}
\]
darstellen lässt.
Setzt man diesen Ausdruck in die Wellengleichung ein, ergibt sich nach Kürzung des Zeitfaktors
\[
\nabla^2 \psi + \frac{\omega^2}{c^2}\psi = 0.
\]
Dies ist die Helmholtz-Gleichung in ihrer Grundform.
Die auftretende Konstante
\[
k = \frac{\omega}{c}
\]
beschreibt hier keine klassische Wellenzahl im Sinne einer 2D-Projektion,
sondern das Verhältnis zwischen Energie und zeitlicher Resonanzfrequenz.
Damit kann man den allgemeinen Resonanzoperator definieren als
\[
\hat{R} = \nabla^2 + k^2,
\]
wobei $\hat{R}\psi = 0$ den stationären Zustand des Raumes beschreibt.
Dieser Operator verknüpft Raum und Zeit zu einer messbaren geometrischen Einheit
und wird in der Energie-Resonanz-Theorie als \emph{Raumresonanzoperator} bezeichnet.

\paragraph{Eigenschaften des Operators.}
Damit $\hat{R}$ eine reale, physikalische Bedeutung besitzt, muss er:
\begin{enumerate}
    \item \textbf{linear} sein, um Überlagerungen (Superposition) zuzulassen,
    \item \textbf{selbstadjungiert} sein, damit die Eigenwerte reell und beobachtbar bleiben,
    \item \textbf{räumlich symmetrisch} sein, um isotrope Resonanzen zu gewährleisten.
\end{enumerate}
Diese Eigenschaften führen zwingend dazu, dass nur reale, messbare Zustände
in der Lösungsmannigfaltigkeit vorkommen – die Theorie bleibt damit
vollständig empirisch anschlussfähig.

\section{Die drei Dimensionen der Realität}

In einem realen Resonanzraum existieren zu jeder Richtung zwei entgegengesetzte Ausbreitungsrichtungen.
Diese Dualität kann als Spiegelung der Energieflüsse verstanden werden.
Damit beschreibt der Laplace-Operator keine Summe unabhängiger Richtungen,
sondern eine gekoppelte Struktur:
\[
\nabla^2 \Psi =
\frac{\partial^2 \Psi}{\partial x_+^2} +
\frac{\partial^2 \Psi}{\partial x_-^2} +
\frac{\partial^2 \Psi}{\partial y_+^2} +
\frac{\partial^2 \Psi}{\partial y_-^2} +
\frac{\partial^2 \Psi}{\partial z_+^2} +
\frac{\partial^2 \Psi}{\partial z_-^2}.
\]
Hierbei stehen $(x_+,x_-)$, $(y_+,y_-)$ und $(z_+,z_-)$
für die dualen Ausbreitungsrichtungen der Raumdimensionen.
Diese Symmetrie beschreibt erstmals ein \emph{physikalisch geschlossenes Raumfeld},
in dem sich Energie als stehende dreidimensionale Welle stabilisiert.

\section{Robin-Randbedingungen als physikalische Kopplung}

An der Grenze eines Resonanzraumes darf Energie weder verloren gehen noch reflektionsfrei entweichen.
Diese Bedingung wird durch die Robin-Randbedingung erfüllt:
\[
\frac{\partial \Psi}{\partial n} + \beta \Psi = 0,
\]
wobei der Parameter $\beta$ die Stärke der Kopplung zwischen Innen- und Außenraum beschreibt.
Der Term $\partial_n \Psi$ steht für den Energiefluss über die Raumgrenze,
$\beta \Psi$ für die Rückwirkung des umgebenden Resonanzfeldes.
Gemeinsam bilden beide Terme die physikalisch reale Stabilisierung eines Resonanzvolumens.

\section{Das entstehende Eigenwertproblem}

Wird der Operator $\hat{R}$ unter den Robin-Bedingungen gelöst,
so entsteht ein Eigenwertproblem der Form
\[
\hat{R}\Psi_i = \lambda_i \Psi_i,
\]
mit
\[
\nabla^2 \Psi_i + k_i^2 \Psi_i = 0, \quad
\frac{\partial \Psi_i}{\partial n} + \beta \Psi_i = 0.
\]
Die Eigenwerte $k_i$ repräsentieren die diskreten Resonanzzustände des Raumes.
Jeder Eigenwert steht für ein stabil existierendes Energieverhältnis,
das im physikalischen Raum beobachtbar sein muss.

\newpage

\subsection{Emergente Symmetrie der dreidimensionalen Randbedingung}

Die duale Robin-Randbedingung erzwingt,
dass jede Welle sich in beiden Richtungen
entlang jeder Raumachse ausbreitet.
Damit entstehen \(2^3 = 8\)
mögliche Überlagerungsrichtungen.
Diese bilden die kleinstmögliche stabile
Resonanzeinheit der dreidimensionalen Geometrie.

Das resultierende 8er-Raster
beschreibt die räumliche Diskretisierung,
die im folgenden Kapitel
zur numerischen Überprüfung herangezogen wird.

\section{Überleitung zum numerischen Experiment}

Das genannte Eigenwertproblem liefert die Grundlage für den empirischen Abgleich.
Durch numerische Lösung der Gleichung für verschiedene Randbedingungen
kann das Spektrum der Eigenwerte bestimmt werden.
Der folgende Schritt besteht darin,
diese berechneten Eigenwerte mit beobachtbaren Größen zu vergleichen
und nach einem konstanten Verhältnis zu suchen,
das alle Skalen miteinander verbindet.

