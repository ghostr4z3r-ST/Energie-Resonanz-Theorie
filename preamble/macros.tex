% Einfache, saubere Makros
\newcommand{\alphastar}{\ensuremath{\alpha^{\ast}}}
\newcommand{\ERTh}{Energie-Resonanz-Theorie}
\newcommand{\helm}{Helmholtz-Gleichung}
\newcommand{\robin}{Robin-Randbedingung}

% Vektor/Laplace etc. (falls physics nicht genutzt wird)
% \newcommand{\laplace}{\nabla^2}

% Einheiten-Kürzel
\newcommand{\GB}{\ensuremath{\,\mathrm{GB}}}
\newcommand{\EB}{\ensuremath{\,\mathrm{EB}}}

% Abstract-Umgebung für scrbook definieren
\providecommand{\abstractname}{Zusammenfassung}
\makeatletter
\newenvironment{abstract}{%
  \cleardoublepage
  \chapter*{\abstractname}%
  \addcontentsline{toc}{chapter}{\abstractname}%
  \markboth{\abstractname}{\abstractname}%
}{\par}
\makeatother

% Lockerere Float-Regeln
\makeatletter
\renewcommand{\topfraction}{0.95}      % Anteil oben: bis 95% Float erlaubt
\renewcommand{\bottomfraction}{0.95}   % Anteil unten: bis 95% Float
\renewcommand{\textfraction}{0.01}     % Mindesttextanteil auf Seite: nur 5%
\renewcommand{\floatpagefraction}{0.5}
\setcounter{topnumber}{3}              % max. Floats oben
\setcounter{bottomnumber}{2}           % max. Floats unten
\setcounter{totalnumber}{5}            % max. Floats gesamt pro Seite
\makeatother

% (optional, für engere Abstände)
\setlength{\textfloatsep}{10pt plus 2pt minus 2pt}
\setlength{\intextsep}{10pt plus 2pt minus 2pt}