
% - Benutzung im Text:
%   \gls{alpha-star}, \Gls{spiralis}, \acrshort{ERT}, \acrlong{ART} etc.

% Haupt-Begriffe

\newglossaryentry{alpha-star}{%
  name={\ensuremath{\alpha^\star}},
  sort={alpha-star},
  text={\ensuremath{\alpha^\star}},
  description={%
    Dimensionsloser Operator der Energie–Zeit–Abweichung zur idealen Periode; in der ERT der \emph{Öffnungswinkel} der realen Spiraldynamik. %
    \ensuremath{\alpha^\star} legt die aperiodische Phasenverschiebung fest und bestimmt damit Resonanz, Skalenkopplung und Wahrnehmungsrahmen.},
  plural={\ensuremath{\alpha^\star}}%
}

\newglossaryentry{spiralis}{%
  name={Spiralis},
  sort={spiralis},
  description={%
    Erweiterung der Sinusfunktion zur \emph{dreidimensionalen, aperiodischen} Felddynamik. %
    Die Spiralis erzeugt aus drei dualen Raumrichtungen ein fraktales, resonantes Feld, dessen Zeitverlauf als geometrische Folge erscheint.},
  plural={Spirales}%
}

\newglossaryentry{aperiodizitaet}{%
  name={Aperiodizität},
  sort={aperiodizitaet},
  description={%
    Abwesenheit exakter Wiederkehr. In der ERT ist Aperiodizität eine essentielle Eigenschaft realer Dynamik: %
    Umlaufbahnen, Energieflüsse und Kopplungen schließen nicht exakt, sondern nähern sich asymptotisch periodischen Mustern an.}%
}

\newglossaryentry{resonanz}{%
  name={Resonanz},
  sort={resonanz},
  description={%
    Überlagerung kohärenter Wellenanteile mit Stabilisierung bestimmter Strukturen (Knoten/Zonen). %
    In der ERT entstehen so hierarchische Kopplungen (Proton $\rightarrow$ H $\rightarrow$ H$_2$ $\rightarrow$ H$^2$O).}%
}

\newglossaryentry{helmholtz}{%
  name={Helmholtz-Gleichung},
  sort={helmholtz},
  description={%
    Partielle Differentialgleichung, die stationäre Wellenfelder beschreibt. %
    In der ERT dient sie als Ausgangspunkt für das Eigenwertproblem im dreidimensionalen Randwert-Setup.}%
}

\newglossaryentry{robin}{%
  name={Robin-Randbedingung},
  sort={robin},
  description={%
    Lineare Randbedingung der Form \ensuremath{a u + b\,\partial_n u = c} an der Grenze des Lösungsgebiets. %
    In der ERT modelliert sie Energiefluss und Balance zwischen Feld und „Außenraum“.}%
}

\newglossaryentry{eigenwert}{%
  name={Eigenwert},
  sort={eigenwert},
  description={%
    Spektralbegriffe des Randwertproblems: Eigenwerte quantisieren zulässige Feldmodi; Eigenfunktionen sind die zugehörigen räumlichen Muster. %
    Der numerische Scan/ Fit der ERT identifiziert ein konsistentes Spektralsignal für \gls{alpha-star}.}%
}

\newglossaryentry{wahrnehmungsrahmen}{%
  name={Wahrnehmungsrahmen},
  sort={wahrnehmungsrahmen},
  description={%
    Effektive, periodische Maske menschlicher Messung und Erfahrung. %
    In der ERT wird Zeit als aus dem Feld emergierende Geometrie verstanden; der Wahrnehmungsrahmen begrenzt Sichtbarkeit (z.\,B. Lichtgeschwindigkeit, EM-Spektrum).}%
}

\newglossaryentry{dualitaet}{%
  name={Dualität (Richtungen)},
  sort={dualitaet},
  description={%
    Jede Raumdimension besitzt zwei Richtungen; Negativität wird nicht benötigt. %
    Die dreifache Dualität erzeugt acht erste Überlagerungspunkte (Ecken eines Würfels) als elementare Resonanzstruktur.}%
}

\newglossaryentry{achter-raster}{%
  name={8er-Raster},
  sort={achter-raster},
  description={%
    Diskretisierung entlang der acht kubischen Richtungen. %
    Dient im ERT-Scan als minimaler symmetrischer Suchraum, in dem Resonanzknoten zuverlässig detektierbar werden.}%
}

\newglossaryentry{hierarchie}{%
  name={Diadische Hierarchie},
  sort={hierarchie},
  description={%
    Skalierung/Einbettung von Resonanzknoten über Faktoren von zwei und acht, wodurch fraktale, selbstähnliche Strukturen über Skalen hinweg entstehen.}%
}

\newglossaryentry{normalisierung}{%
  name={Normalisierung},
  sort={normalisierung},
  description={%
    Abbildung der berechneten Felder auf \([0,1]\) zur Visualisierung. %
    Der Punkt \mbox{\(0{,}5\)} markiert die Wahrnehmungsnull (Raum), während \(\,1-\alpha^\star/10\,\) die sichtbare Informationsgrenze setzt.}%
}

\newglossaryentry{em-colormap}{%
  name={EM-Colormap (ERT)},
  sort={em-colormap},
  description={%
    Farbcodierung, die das sichtbare Spektrum (Violett\,$\to$\,Weiß) in Einklang mit \gls{alpha-star} abbildet, inkl.\ UV/IR-Interpretation als „Rand“-Informationen des Wahrnehmungsfensters.}%
}

\newglossaryentry{vti}{%
  name={VTI-Volumen},
  sort={vti},
  description={%
    VTK ImageData-Dateiformat für reguläre Gitter (ParaView-kompatibel), genutzt zur Speicherung der berechneten ERT-Felder (z.\,B. Proton, H, H$_2$, O, H$^2$O).}%
}

\newglossaryentry{paraview}{%
  name={ParaView},
  sort={paraview},
  description={%
    Open-Source-Werkzeug zur Volumenvisualisierung. %
    In der ERT dient ParaView als experimentelle Bühne zur Sichtbarmachung der Spiralis-Felder und ihrer Hierarchien.}%
}

\newglossaryentry{residuum}{%
  name={Residuum/Residuen-Fit},
  sort={residuum},
  description={%
    Maß für Abweichungen zwischen Modell und Daten; in der ERT zur Verifikation von \gls{alpha-star} über Skalen (atomar, kosmologisch) eingesetzt.}%
}

\newglossaryentry{wahrnehmungsfilter}{%
  name={Wahrnehmungsfilter (UV/IR)},
  sort={wahrnehmungsfilter},
  description={%
    Reduktion der Opazität außerhalb des sichtbaren Spektrums (UV, IR), um die strukturtragenden Bereiche der Felder klar erkennbar zu machen.}%
}

\newglossaryentry{zeit-als-geometrie}{%
  name={Zeit als Geometrie},
  sort={zeit-als-geometrie},
  description={%
    ERT-Kerngedanke: Zeit ist keine unabhängige Koordinate, sondern die emergente Folge des aperiodischen Spiralverlaufs im dreidimensionalen Feld.}%
}

% ---------- Akronyme ----------

\newacronym{ERT}{ERT}{Energie-Resonanz-Theorie}
\newacronym{ART}{ART}{Allgemeine Relativitätstheorie}
\newacronym{EM}{EM}{Elektromagnetisch / Elektromagnetisches Spektrum}
\newacronym{CMB}{CMB}{Kosmische Mikrowellen-Hintergrundstrahlung}
\newacronym{VTK}{VTK}{Visualization Toolkit (Dateiformate/IO)}
